\problemname{Tv-kiggeri}
Børges venner elsker tv-serier og plejer at diskutere dem på deres fødselsdagsfester.
Børge føler sig ofte udenfor, fordi han ikke har set de samme serier som de andre.

Børge er inviteret til fødselsdagsfest på visse dage og vil med til dem alle.
Han ved i forvejen, hvilke tv-serier som bliver diskuteret til hvilke fester, og han vil derfor se de pågældende serier færdigt for at kunne snakke med.

Børge vil ikke se tv mere end ti timer per dag, og han har ikke tid til at se tv på de dage, hvor han skal på fest.
Børge vil ikke se fjernsyn mere end ti timer per dag, og han har ikke tid til at se fjernsyn på de dage, hvor han skal på fest.

Han kan når som helst tage en pause fra en tv-serie og fortsætte med at se videre en anden dag, men når han er på en fest, hvor serien bliver diskuteret, skal han have set den færdig.
Hjælp Børge med at planlægge sit tv-forbrug.

\section*{Indlæsning}
På første linje står de to heltal $n$ og $k$ ($1 \leq n,k \leq 10^5$), som angiver antallet af fester og antallet af tv-serier.
Tv-serierne er nummererede fra $1$ til $k$.

På næste linje står $k$ heltal, hvoraf det $i$te tal angiver længden af tv-serie nummer~$i$ målt i timer.
Ingen serie er længere end $10^6$ timer.

De følgende $n$ linjer beskriver festerne i rækkefølge.
Linje~$i$ begynder med to heltal $1 \leq d_i \leq 10^5$ og $c_i$, som angiver hvilken dag festen finder sted og antallet af tv-serier som vil blive diskuteret.
Derefter følger $c_i$ forskellige heltal på samme linje, nemlig de tv-serier der diskuteres til festen.
Summen af alle $c_i$ overstiger ikke $10^5$.

Børge er højst inviteret til en fest per dag. 
Nu er det morgen på dag~$0$, og Børge skal altså ikke til nogen fest i dag.

\section*{Udskrift}
Udskriv »\texttt{Ja}«, hvis det er muligt at se alle tv-serier færdigt inden de fester, hvor de vil blive diskuteret.
Udskriv »\texttt{Nej}«, hvis det ikke er muligt.

\section*{Pointgivning}
Din løsning bliver kørt på en række testfaldsgrupper.
For at opnå point for en gruppe, skal din løsning klare samtlige testfald i gruppen.

\noindent
\begin{tabular}{ lll}
Gruppe & Point & Begrænsninger \\ \hline
$1$   & $20$       & $n \leq 50$, $k \leq 50$ \\
$1$   & $40$       & $n \leq 1000$ \\
$2$   & $40$       & Ingen yderligere begrænsninger.
\end{tabular}

\section*{Forklaring af eksempel 1}
Næste fest er om to dage, og der vil den 20 timer lange tv-serie~2 blive diskuteret.
Det kan Børge præcis nå ved at se fjernsyn i 10 timer hver dag i to dage.
Derefter følger en fridag, hvor Børge akkurat når at se serie~3 og 4, som hver tager fem timer.
På dag~5 kan han nøjes med at kigge på serie~1, fordi han allerede har set serie~2.
