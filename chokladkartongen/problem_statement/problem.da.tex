\problemname{Chokoladeæsken}
Alex elsker chokolade.
Han har derfor altid en en æske i spisekammeret, som han sommetider smugspiser fra.
 år æsken er tom, køber han en ny og lader som ingenting.
Alex’ kæreste Kim aner dog uråd -- hvorfor bliver den der æske egentlig aldrig tom? -- og er begyndt at holde øje med antallet af de resterende chokoladestykker. 

Skriv et program, der på grundlag af Kims observationer udregner det mindste antal nye æsker, som Alex må have købt i løbet af perioden.

\section*{Indlæsning}

På første linje står et heltal $N \le 100$, antallet af observationer.
Derefter følger en linje med $N$ heltal $a_1$, $\lodts$, $a_N$, antallet af chokoladestykker i æsken for hver observation.
Der gælder $1\leq a_i\leq 100$ for alle $i$. 

\section*{Udskrift}
Skriv en enkelt linje indeholdende et heltal: det mindste antal nye chokoladeæsker, som Alex beviseligt må have købt i løbet af perioden.
